\section{Introduction}
Car Sharing is a fast-growing model in the mobility sector. Nearly in every big city in the world, a car-sharing service is offered. It allows members to use private cars without owning an own car \cite{Bardhi_1}. This reduces the costs for the customers and makes the mobility sector more sustainable. People can profit from the benefits of using an own car without bearing the costs of owning one. In 1987, the first car-sharing company was launched in Switzerland. Over the last 30 years, this has developed into a model that is used all over the world \cite{Sus_1}. Generally, car-sharing can be split into two types: reservation-based and free-floating. In the case of reservation-based, customers have to reserve the vehicles before using them, and in the case of free-floating-base, customers can pick up any available car and can directly use it. The rentals of the cars are split into two categories: one-way and round-trip rentals. The difference is that for round trip rentals, the customer has to rent and return the car at the same location. For one-way rentals, rent and return location can be different \cite{Yu_1}.\\
The main impact area of car sharing are transportation, environment, land use, and society \cite{Sus_1}. It brings advantages to all these areas. Some of the major advantages are the following:
\begin{itemize}
\item The reduction of personal transportation costs, because people only pay for the time they rent the car.\cite{Sus_2}
\item The reduction of total traffic, because car-sharing users plan their trips more efficiently. \cite{Cooper_1}
\item A side effect of the traffic reduction is an increase in traffic safety because there are fewer accidents. \cite{Cooper_1}
\item Carsharing leads to a reduction in vehicle ownership and the total number of vehicles.\cite{MACEDO2017731}
\item Many car-sharing service providers use low emission vehicles, which are more sustainable, and people who get a car-sharing membership report to raise a higher awareness for the environment after they get the membership.\cite{Sus_2,lane_1}
\end{itemize}
A big problem that car-sharing service providers face is optimal vehicle allocation. The object is to optimize the vehicle allocation in time and space and maximize the provider's profit. Supply and demand are not always equal in some areas, so the service provider has to reallocate empty vehicles to achieve optimal use of all resources. There are different approaches to solve the problem Fan\cite{Wei_1} discussed a multistage stochastic linear programming model with uncertain demand for one-way reservation-based service. In this survey, the model gets expanded with penalty costs for unfulfilled demand to get a more realistic model.\\
The rest of the survey is structured as follows. First, the problem and assumptions are described. Then the generation of the scenario tree of the uncertain demand is shown. After this, the stochastic programming model is formulated. Based on that, a small case study is presented and solved. The last part is the conclusion, where the results are summarized, and future research opportunities are identified.