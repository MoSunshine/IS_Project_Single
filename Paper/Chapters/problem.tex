\section{Problem Description}
The dynamic vehicle allocation problem can be described as follows. A car-sharing service provider has a given number of vehicles and wants to offer his service at a given number of locations in the neighborhood for a time horizon. The customer can rent a car and drive from one place to the same or another location. The service provider can reallocate all vehicles that customers do not use. At the beginning of a time horizon, the service provider knows the number of customers who want to rent a car and drive from one specific to another specific location. He has to decide if he wants to accept each customer's reservation or if he wants to reject the reservation, e.g., if the reservation is unprofitable or if he is unable to fulfill the total demand. Also, he has to decide how to allocate the vehicles, which the customers do not use. For every accepted reservation, he gets a revenue, for every rejected reservation, he gets penalty costs, and for every reallocation, he has to pay a given price.
The car-sharing service manager has to decide at the beginning of every period how many reservations he wants to accept, how many he wants to reject, and how he wants to reallocate the vehicles.
The manager only knows the demand for the current period, but not for the future periods, so the demand is uncertain for him.\\
He aims to maximize the total profit over N periods by using independent random future demands with a known distribution. The increasing level of future uncertainty makes this problem a difficult stochastic problem. To model the problem, the following assumptions are made: (1) The time is measured in intervals or periods (in this case one day); in each period, every vehicle is assigned to a customer, is moved empty to another location or stays at the location. (2) The customers make reservations for the next period at the end of each period, so the car-sharing service provider knows the demand at the beginning of each period. (3) The vehicles are picked up at a specific location. (4) The cars are dropped off at any car-sharing location at a particular time the following day. (5) Customers can not share vehicles to fulfill their demand during a period, so each customer needs one car for the whole period two satisfy his demand. (6) A journey between two locations lasts one day. (7) All unfulfilled demands during a period are lost and penalized with costs. (8) All cars are available in the first period. (9) The mean values and the distribution for the demand between two locations are known, and it is assumed to be discrete.\\
We can now formulate this as a profit-maximization problem and develop a stochastic model.